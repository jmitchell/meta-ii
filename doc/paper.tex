\documentclass[notitlepage,twocolumn]{report}
\usepackage[left=1in, right=1in, top=1in, bottom=1in]{geometry}
\usepackage{endnotes}
\usepackage{graphicx}
\usepackage{caption}
\usepackage{subcaption}

\title{META II \\ \large A Syntax-Oriented Compiler Writing Language}
\author{D. V. Schorre\footnote{Transcribed from http://www.ibm-1401.info/Meta-II-schorre.pdf in January 2017 by Jacob Mithcell.}}
\date{}

\begin{document}
\twocolumn
\maketitle


META II is a compiler writing language which consists of syntax
equations resembling Backus normal form and into which instructions to
output assembly commands are inserted. Compilers have been written in
this language for VALGOL I and VALGOL II. The former is a simple
algebraic language designed for the purpose of illustrating META
II. The latter contains a fairly large subset of ALGOL 60.

The method of writing compilers which is given in detail in the paper
may be explained briefly as follows. Each syntax equation is
translated into a recursive subroutine which tests the input string
for a particular phrase structure, and deletes it if found. Backup is
avoided by the extensive use of factoring in the syntax equations. For
each source language, an interpreter is written and programs are
compiled into that interpretive language.

META II is not intended as a standard language which everyone will use
to write compilers. Rather, it is an example of a simple working
language which can give one a good start in designing a
compiler-writing compiler suited to his own needs. Indeed, the META II
compiler is written in its own language, thus lending itself to
modification.

\section*{History}

The basic ideas behind META II were described in a series of three
papers by Schmidt,\endnote{TODO} Metcalf,\endnote{TODO} and
Schorre.\endnote{TODO} These papers were presented at the 1963
National A.C.M. Convention in Denver, and represented the activity of
the Working Group on Syntax-Directed Compilers of Los Angeles
SIGPLAN. The methods used by that group are similar to those of
Glennie and Conway, but differ in one important respect. Both of these
researchers expressed syntax in the form of diagrams, which they
subsequently coded for use on a computer. In the case of META II, the
syntax is input to the computer in a notation resembling Backus normal
form. The method of syntax analysis discussed in this paper is
entirely different from the one used by Irons\endnote{TODO} and
Bastian.\endnote{TODO} All of these methods can be traced back to the
mathematical study of natural languages, as described by
Chomsky.\endnote{TODO}

\section*{Syntax Notation}

The notation used here is similar to the meta language of the ALGOL 60
report. Probably the main difference is that this notation can be
keypunched. \underline{Symbols} in the target language are represented
as strings of characters, surrounded by
quotes. \underline{Metalinguistic variables} have the same form as
identifiers in ALGOL, viz., a letter followed by a sequence of letters
or digits. Items are written consecutively to indicate
\underline{concatenation} and separated by a slash to indicate
alternation. Each equation ends with a semicolon which, due to
keypunch limitations, is represented by a period followed by a
comma. An example of a syntax equation is:

\begin{verbatim}
LOGICALVALUE = '.TRUE' / '.FALSE' .,
\end{verbatim}

In the versions of ALGOL described in this paper the symbols which are
usually printed in boldface type will begin with periods, for example:

\begin{verbatim}
.PROCEDURE .TRUE .IF
\end{verbatim}

To indicate that a syntactic element is \underline{optional}, it may
be put in alternation with the word \texttt{.EMPTY}. For example:

\begin{verbatim}
SUBSECONDARY = '*' PRIMARY / .EMPTY .,
SECONDARY = PRIMARY SUBSECONDARY .,
\end{verbatim}

By factoring, these two equations can be written as a single equation.

\begin{verbatim}
SECONDARY = PRIMARY('*' PRIMARY / .EMPTY) .,
\end{verbatim}

Built into the META II language is the ability to recognize three
basic symbols which are:

\begin{enumerate}
\item Identifiers -- represented by \texttt{.ID},
\item Strings -- represented by \texttt{.STRING},
\item Numbers -- represented by \texttt{.NUMBER}.
\end{enumerate}

The definition of identifier is the same in META II as in ALGOL, viz.,
a letter followed by a sequence of letters or digits. The definition
of a string is changed because of the limited character set available
on the usual keypunch. In ALGOL, strings are surrounded by opening and
closing quotation marks, making it possible to have quotes within a
string. The single quotation mark on the keypunch is unique, imposing
the restriction that a string in quotes can contain no other quotation
marks.

The definition of number has been radically changed. The reason for
this is to cut down on the space required by the machine subroutine
which recognizes numbers. A number is considered to be a string of
digits which may include imbedded periods, but may not begin or end
with a period; moreover, periods may not be adjacent. The use of the
subscript 10 has been eliminated.

Now we have enough of the syntax defining features of the META II
language so that we can consider a simple example in some detail.

The example given here is a set of four syntax equations for defining
a very limited set of algebraic expressions. The two operators,
addition and multiplication, will be represented by \texttt{+} and
\texttt{*} respectively. Multiplication takes precedence over
addition; otherwise precedence is indicated by parentheses. Some
examples are:

\begin{verbatim}
A
A + B
A + B * C
(A + B) * C
\end{verbatim}

The syntax equations which define this class of expressions are as
follows:

\begin{verbatim}
EX3 = .ID / '(' EX1 ')' .,
EX2 = EX3 ('*' EX2 / .EMPTY) .,
EX1 = EX2 ('+' EX1 / .EMPTY) .,
\end{verbatim}

\texttt{EX} is an abbreviation for expression. The last equation,
which defines an expression of order 1, is considered the main
equation. An expression of order 3 is defined as an identifier or an
open parenthesis followed by an expression of order 1 followed by a
closed parenthesis. An expression of order 2 is defined as an
expression of order 3, which may be followed by a star which is
followed by an expression of order 2. An expression of order 1 is
defined as an expression of order 2, which may be followed by a plus
which is followed by an expression of order 1.

Although sequences can be defined recursively, it is more convenient
and efficient to have a special operator for this purpose. For
example, we can define a sequence of the letter \texttt{A} as follows:

\begin{verbatim}
SEQA = $ 'A' .,
\end{verbatim}

The equations given previously are rewritten using the sequence
operator as follows:

\begin{verbatim}
EX3 = .ID / '(' EX1 ')' .,
EX2 = EX3 $ ('*' EX3) .,
EX1 = EX2 $ ('+' EX2) .,
\end{verbatim}


\section*{Output}

Up to this point we have considered the notation in META II which
describes object language syntax. To produce a compiler, output
commands are inserted into the syntax equations. Output from a
compiler written in META II is always in an assembly language, but not
in the assembly language for 1401. It is for an interpreter, such as
the interpreter I call the META II machine, which is used for all
compilers, or the interpreters I call the VALGOL I and VALGOL II
machines, which obviously are used with their respective source
languages. Each machine requires its own assembler, but the main
difference between assemblers is the operation code table. Constant
codes and declarations may also be different. These assemblers all
have the same format, which is shown below.

\begin{verbatim}
LABEL      CODE      ADDRESS
1-   -6    8-  -10   12-             -70
\end{verbatim}

An assembly language record contains either a label or an op code of
up to 3 characters, but never both. A label begins in column 1 and may
extend as far as column 70. If a record contains an op code, then
column 1 must be blank. Thus labels may be any length and are not
attached to instructions, but occur between instructions.

To produce output beginning in the op code field we write
\texttt{.OUT} and then surround the information to be reproduced with
parentheses. A string is used for literal output and an asterisk to
output the special symbol just found in the input. This is illustrated
as follows:

\begin{verbatim}
EX3 = .ID .OUT('LD ' *) / '(' EX1 ')' .,
EX2 = EX3 $ ('*' EX3 .OUT('MLT')) .,
EX1 = EX2 $ ('+' EX2 .OUT('ADD')) .,
\end{verbatim}

To cause output in the label field we write \texttt{.LABEL} followed
by the item to be output. For example, if we want to test for an
identifier and output it in the label field we write:

\begin{verbatim}
.ID .LABEL *
\end{verbatim}

The META II compiler can generate labels of the form A01, A02, A03,
\ldots A99, B01, \ldots. To cause such a label to be generated, one
uses \texttt{*1} or \texttt{*2}. The first time \texttt{*1} is
referred to in any syntax equation, a label will be generated and
assigned to it. This same label is output whenever \texttt{*1} is
referred to within that execution of the equation. The symbol
\texttt{*2} works in the same way. Thus a maximum of two different
labels may be generated for each execution of any equation. Repeated
executions, whether recursive or externally initiated, result in a
continued sequence of generated labels. Thus all syntax equations
contribute to the one sequence. A typical example in which labels are
generated for branch commands is now given.

\begin{verbatim}
IFSTATEMENT = '.IF' EXP '.THEN' .OUT('BFP' *1)
     STATEMENT '.ELSE' .OUT('B ' *2) .LABEL *1
     STATEMENT .LABEL *2 .,
\end{verbatim}

The op codes \texttt{BFP} and \texttt{B} are orders of the VALGOL I
machine, and stand for ``branch false and pop'' and ``branch''
respectively. The equation also contians references to two other
equations which are not explicitly given, viz., \texttt{EXP} and
\texttt{STATEMENT}.


\section*{VALGOL I -- A Simple Compiler Written in META II}

Now we are ready for an example of a compiler written in META
II. VALGOL I is an extremely simple language, based on ALGOL 60, which
has been designed to illustrate the META II compiler.

The basic information about VALGOL I is given in figure 1 (the VALGOL
I compiler written in META II) and figure 2 (order list of the VALGOL
I machine). A sample program is given in figure 3. After each line of
the program, the VALGOL I commands which the compiler produces from
that line are shown, as well as the absolute interpretive language
produced by the assembler. Figure 4 is output from the sample
program. Let us study the compiler writtein in META II (figure 1) in
more detail.

The identifier \texttt{PROGRAM} on the first line indicates that this
is the main equation, and that control goes there first. The equation
for \texttt{PRIMARY} is similar to that of \texttt{EX3} in our
previous example, but here numbers are recognized and reproduced with
a ``load literal'' command. \texttt{TERM} is what was previously
\texttt{EX2}; and \texttt{EXP1} what was previously \texttt{EX1}
except for recognizing minus for subtraction. The equation
\texttt{EXP} defines the relational operator ``equal'', which produces
a value of 0 or 1 by making a comparison. Notice that this is handled
just like the arithmetic operators but with lower precedence. The
conditional branch commands, ``branch true and pop'' and ``branch
false and pop'', which are produced by the equations definining
\texttt{UNTILST} and \texttt{CONDITIONALST} respectively, will test
the top item in the stack and branch accordingly.

The ``assignment statement'' defined by the equation for
\texttt{ASSIGNST} is reversed from the convention in ALGOL 60, i.e.,
the location into which the computed value is to be stored is on the
right. Notice also that the equal sign is used for the assignment
statement and that period equal (\texttt{.=}) is used for the relation
discussed above. This is because assignment statements are more
numerous in typical programs than equal compares, and so the simpler
representation is chosen for the more frequently occurring.

The omission of statement labels from the VALGOL I and VALGOL II seems
strange to most programmers. This was not done because of any
difficulty in their implementation, but because of a dislike for
statement labels on the part of the author. I have programmed for
several years without using a single label, so I know that they are
superfluous from a practical, as well as from a theoretical,
standpoint. Nevertheless, it would be too much of a digression to try
to justify this point here. The ``until statement'' has been added to
facilitate writing loops without labels.

The ``conditional'' statement is similar to one in ALGOL 60, but here
the ``else'' clause is required.

The equation for ``input/output'', \texttt{IOST}, involves two
commands, ``edit'' and ``print''. The words \texttt{EDIT} and
\texttt{PRINT} do no begin with periods so that they will look like
subroutines written in code. ``\texttt{EDIT}'' copies the given string
into the print area, with the first character in the print position
which is computed from the given expression. ``\texttt{PRINT}'' will
print the current contents of the print area and then clear it to
blanks. Giving a print command without previous edit commands results
in writing a blank line.

\texttt{IDSEQ1} and \texttt{IDSEQ} are given to simplify the syntax
equation for \texttt{DEC} (declaration). Notice in the definition of
\texttt{DEC} that a branch is given around the data.

From the definition of \texttt{BLOCK} it can be seen that what is
considered a compound statement in ALGOL 60 is, in VALGOL I, a special
case of a block which has no declaration.

In the definition of a statement, the test for an \texttt{IOST}
precedes that for an \texttt{ASSIGNST}. This is necessary, because if
ithis were not done the words \texttt{PRINT} and \texttt{EDIT} would
be mistaken as identifiers and the compiler would try to translate
``input/output'' statements as if they were ``assignment'' statements.

Notice that a \texttt{PROGRAM} is a block and that a standard set of
commands is output after each program. The ``halt'' command causes the
machine to stop on reaching the end of the outermost block, which is
the program. The operation code \texttt{SP} is generated after the
``halt'' command. This is a completely 1401-oriented code, which
serves to set a word mark at the end of the program. It would not be
used if VALGOL I were implemented on a fixed word-length machine.


\section*{How the META II Compiler Was Written}

Now we come to the most interesting part of this project, and consider
how the META II compiler was written in its own language. The
interpreter called the META II machine is not a much longer 1401
program than the VALGOL I machine. The syntax equations for META II
(figure 5) are fewer in number than those for the VALGOL I machine
(figure 1).

The META II compiler, which is an interpretive program for the META II
machine, takes the syntax equations given in figure 5 and produces an
assembly language version of this same interpretive program. Of
course, to get this started, I had to write the first compiler-writing
compiler by hand. After the program was running, it could produce the
same program as written by hand. Someone always asks if the compiler
really produced exactly the program I had written by hand and I have
to say that it was ``almost'' the same program. I followed the syntax
equations and tried to write just what the compiler was going to
produce. Unfortunately I forgot one of the redundant instructions, so
the results were not quite the same. Of course, when the first
machine-produced compiler compiled itself the second time, it
reproduced itself exactly.

The compiler originally written by hand was for a language called META
I. This was used to implement the improved compiler for META
II. Sometimes when I wanted to change the metalanguage, I could not
describe the new metalanguage directly in the current
metalanguage. Then an intermediate language was created--one which
could be described in the currently language and in which the new
language could be described. I thought that it might sometimes be
necessary to modify the assembly language output, but it seems that it
is always possible to avoid this with the intermediate language.

The order list of the META II machine is given in figure 6.

All subroutines in META II programs are recursive. When the program
enters a subroutine a stack is pushed down by three cells. One cell is
for the exit address and the other two are for labels which may be
generated during the execution of the subroutine. There is a switch
which may be set or reset by the instructions which refer to the input
string, and this is the switch referred to by the conditional branch
commands.

The first thing in any META II machine program is the address of the
first instruction. During the initialization for the interpreter, this
address is placed into the instruction counter.


\section*{VALGOL II Written in META II}

VALGOL II is an expansion of VALGOL I, and serves as an illustrction
of a fairly elaborate programming language implemented in the META II
system. There are several features in the VALGOL II machine which were
not present in the VALGOL I machine, and which require some
explanation. In the VALGOL II machine, addresses as well as numbers
are put in the stack. They are marked appropriately so they can be
distinguished at execution time.

The main reason that addresses are allowed in the stack is that, in
the case of a subscripted variable, an address is the result of a
computation. In an assignment statement each left member is compiled
into a sequence of code which leaves an address on top of the
stack. This is done for simple variables as well as subscripted
variables, because the philosophy of this compiler writing system has
been to compile everything in the most general way. A variable, simple
or subscripted, is always compiled into a sequence of instructions
which leaves an address on top of the stack. The address is not
replaced by its contents until the actual value of the variable is
needed, as in an arithmetic expression.

A formal parameter of a procedure is stored either as an address or as
a value which is computed when the procedure is called. It is up to
the load command to go through any number of indirect address in order
place the address of a number on the stack. An argument of a procedure
is always an algebraic expression. In case this expression is a
variable, the value of the formal parameter will be an address
computed upon entering the procedure; otherwise, the value of the
formal parameter will be a number computed upon entering the
procedure.

The operation of a load command is now described. It causes the given
address to be put on top of the stack. If the content of this top item
happens to be another address, then it is replaced by that
address. This continues until the top item on the stack is the address
of something which is not an address. This allows for formal
parameters to refer to other formal parameters at any depth.

No distinction is made between integer and real numbers. An integer is
just a real number whose digits right of the decimal point are
zero. Variables initially have a value called ``undefined'', and any
attempt to use this value will be indicated as an error.

An assignment statement consists of any number of left parts followed
by a right part. For each left part there is compiled a sequence of
commands which puts an address on top of the stack. The right part is
compiled into a sequence of instructions which leaves on top of the
stack either a number of the address of a number. Following the
instruction for the right part there is a sequence of store commands,
one for each left part. The first command of this sequence is ``save
and store'', and the rest are ``plain'' store commands. The ``save and
store'' puts the number which is on top of the stack (or which is
referred to by the address on top of the stack) into a register called
\texttt{SAVE}. It then stores the contents of \texttt{SAVE} in the
address which is held in the next to top position of the
stack. Finally it pops the top two items, which it has used, out of
the stack. The number, however, remains in \texttt{SAVE} for use by
the following store commands. Most assignment statements have only one
part left, so ``plain'' store commands are seldom produced, with the
result that the number put in \texttt{SAVE} is seldom used again.

% TODO: Illustration 1
% TODO: Illustration 2

The method for calling a procedure can be explained by reference to
illustrations 1 and 2. The arguments which are in the stack are moved
to their place at the top of the procedure. If the number of arguments
in the stack does not correspond to the number of arguments in the
procedure, an error is indicated. The ``flag'' in the stack works like
this. In the VALGOL II machine there is a flag register. To set a flag
in the stack, the contents of this register is put on top of the
stack, then the address of the word above the top of the stack is put
into the flag register. Initially, and whenever there are no flags in
the stack, the flag register contains blanks. At other times it
contains the address of the word in the stack which is just above the
uppermost flag. Just before a call instruction is executed, the flag
register contians the address of the word in the stack which is two
above the word containing the address of the procedure to be
executed. The call instruction picks up the arguments from the stack,
beginning with the one stored just above the flag, and continuing to
the top of the stack. Arguments are moved into the appropriate places
at the top of the procedure being called. An error message is given if
the number of arguments in the stack does not correspond to the number
of places in the procedure. Finally the old flag address, which is
just below the procedure address in the stack, is put in the flag
register. The exit address replaces the address of the procedure in
the stack, and all the arguments, as well as the flag, are popped
out. There are just two op codes which affect the flag register. The
code ``load flag'' puts a flag into the stack, and the code ``call''
takes one out.

The library function ``\texttt{WHOLE}'' truncates a real number. It
does not convert a real number to an integer, because no distinction
is made between them. It is substituted for the recommended function
``\texttt{ENTIER}'' primarily because truncation takes fewer machines
instructions to implement. Also, truncation seems to be used more
frequently. The procedure \texttt{ENTIER} can be defined in VALGOL II
as follows:

\begin{verbatim}
.PROCEDURE ENTER(X) .,
   .IF O .L= X .THEN WHOLE (X) .ELSE
   .IF WHOLE(X) = X .THEN X .ELSE
   WHOLE(X) -1
\end{verbatim}

The ``for statement'' in VALGOL II is not the same as it is in
ALGOL. Exactly one list element is required. The ``step .. until''
portion of the element is mandatory, but the ``while'' portion may be
added to terminate the loop immediately upon some condition. The
iteration continues so long as the value of the variable is less than
or equal to the maximum, irrespective of the sign of the
increment. Illustration 3 is an example of a typical ``for
statement''. A flow chart of this statement is given in illustration
4.

% TODO: Illustration 3
% TODO: Illustration 4

Figure 7 is a listing of the VALGOL II compiler written in META
II. Figure 8 gives the or-order list of the VALGOL II machine. A
sample program to take the determinant is given in figure 9.


\section*{Backup vs. No Backup}

Suppose that, upon entry to a recursive subroutine, which represents
some syntax equation, the position of the input and output are
saved. When some non-first term of a component is not found, the
compiler does not have to stop with an indication of a syntax
error. It can back-up the input and output and return false. The
advantages of backup are as follows:

\begin{enumerate}
\item It is possible to describe languages, using backup, which cannot
  be described without backup.
\item Even for a language which can be described without backup, the
  syntax equations can often be simplified when backup is allowed.
\end{enumerate}

The advantages claimed for non-backup are as follows:

\begin{enumerate}
\item Syntax analysis is faster.
\item It is possible to tell whether syntax equations will work just
  by examining them, without following through numerous examples.
\end{enumerate}

The fact that rather sophisticated languages such as ALGOL and COBOL
can be implemented without backup is pointed out by various people,
including Conway,\endnote{TODO} and they are aware of the speed
advantages of so doing. I have seen no mention of the second advantage
of no-backup, so I will explain this in more detail.

Basically one writes alternatives in which each term begins with a
different symbol. Then it is not possible for the compiler to go down
the wrong path. This is made more complicated because of the use of
``\texttt{.EMPTY}''. An optional item can never be followed by
something that begins with the same symbol it begins with.

The method described above is not the only way in which backup can be
handled. Variations are worth considering, as a way may be found to
have the advantages of both backup and no-backup.


\section*{Further Development of META Languages}

As mentioned earlier, META II is not presented as a standard language,
but as a point of departure from which a user may develop his own META
language. The term ``META Language,'' with ``META'' in capital
letters, is used to denote any compiler-writing language so developed.

The language which Schmidt\endnote{TODO} implemented on the PDP-1 was
based on META I. He has now implemented an improved version of this
language for a Beckman machine.

Rutman\endnote{TODO} has implemented LOGIK, a compiler for bit-time
simulation, on the 7090. He uses a META language to compile Boolean
expressions into efficient machine code. Schneider and
Johnson\endnote{TODO} have implemented META 3 on the IBM 7094, with
the goal of producing an ALGOL compiler which generates efficient
machine code. They are planning a META language which will be suitable
for any block structured language. To this compiler-writing language
they give the name META 4 (pronounced metaphor).


\theendnotes

\onecolumn

\begin{figure}
  \caption{The VALGOL I compiler written in META II Language}
\begin{verbatim}
.SYNTAX PROGRAM

PRIMARY = .ID .OUT('LD ' *) /
     .NUMBER .OUT('LDL' *) /
     '(' EXP ')' .,

TERM = PRIMARY $('*' PRIMARY .OUT('MLT') ) .,

EXP1 = TERM $('+' TERM .OUT('ADD') /
     '-' TERM .OUT('SUB') ) .,

EXP = EXP1 ( '.=' EXP1 .OUT('EQU') / .EMPTY) .,

ASSIGNST = EXP '=' .ID .OUT('ST ' *) .,

UNTILST = '.UNTIL' .LABEL *1 EXP '.DO' .OUT('BTP' *2)
     ST .OUT('B  ' *1) .LABEL *2 .,

CONDITIONALST = '.IF' EXP '.THEN' .OUT('BFP' *1)
     ST '.ELSE' .OUT('B  ' *2) .LABEL *1
     ST .LABEL *2 .,

IOST = 'EDIT' '(' EXP ',' .STRING
     .OUT('EDT' *) ')' /
     'PRINT' .OUT('PNT') .,

IDSEQ1 = .ID .LABEL * .OUT('BLK 1') .,

IDSEQ = IDSEQ1 $(',' IDSEQ1) .,

DEC = '.REAL' .OUT('B  ' *1) IDSEQ .LABEL *1 .,

BLOCK = '.BEGIN' (DEC '.,' / .EMPTY)
     ST $('.,' ST) '.END' .,

ST = IOST / ASSIGNST / UNTILST /
     CONDITIONALST / BLOCK .,

PROGRAM = BLOCK .OUT('HLT')
     .OUT('SP  1') .OUT('END') .,

.END
\end{verbatim}
\end{figure}

\begin{figure}
\caption{Order List of the VALGOL I Machine}
\begin{center}
MACHINE CODES
\end{center}
\begin{tabular}{lllp{7cm}}
  LD & AAA & LOAD & Put the contents of the address AAA on top of the
                    stack. \\
  LDL & NUMBER & LOAD LITERAL & Put the given number on top of the
                                stack. \\
  ST & AAA & STORE & Store the number which is on top of the stack
                     into the address AAA and pop up the stack. \\
  ADD & & ADD & Replace the two numbers which are on top of the stack
                with their sum. \\
  SUB & & SUBTRACT & Subtract the number which is on top of the stack
                     from the number which is next to the top, then
                     replace them by this difference. \\
  MLT & & MULTIPLY & Replace the two numbers which are on top of the
                     stack with their product. \\
  EQU & & EQUAL & Compare the two numbers on top of the stack. Replace
                  them by the integer 1, if they are equal, or by the
                  integer 0, if they are unequal. \\
  B & AAA & BRANCH & Branch to the address AAA. \\
  BFP & AAA & BRANCH FALSE AND POP & Branch to the address AAA if the
                                     top term in the stack is the
                                     integer 0. Otherwise, continue in
                                     sequence. In either case, pop up
                                     the stack. \\
  BTP & AAA & BRANCH TRUE AND POP & Branch to the address AAA if the
                                    top term in the stack is not the
                                    integer 0. Otherwise, continue in
                                    sequence. In either case, pop up
                                    the stack. \\
  EDT & STRING & EDIT & Round the number which is on top of the stack
                        to the nearest integer n. Move the given
                        string into the print area so that its first
                        character falls on print position n. In case
                        this would cause characters to fall outside
                        the print area, no movement takes place. \\
  PNT & & PRINT & Print a line, then space and clear the print
                  area. \\
  HLT & & HALT & Halt. \\
\end{tabular}
\begin{center}
CONSTANT AND CONTROL CODES
\end{center}
\begin{tabular}{lllp{7cm}}
  SP & N & SPACE & N = 1--9. Constant code producing N blank
                   spaces. \\
  BLK & NNN & BLOCK & Produces a block of NNN eight character
                      words. \\
  END & & END & Denotes the end of the program. \\
\end{tabular}
\end{figure}


\begin{figure}
\caption{A program as compiled for the VALGOL I machine}
\label{valgol-1-example}
\begin{verbatim}
.BEGIN
.REAL X ., 0 = X .,
              B   A01                             0000 G 0012
       X                                          0004
              BLK 001                             0004
       A01                                        0012
              LDL 0                               0012 A
              ST  X                               0021 B 0004
.UNTIL X .= 3 .DO .BEGIN
       A02                                        0025
              LD  X                               0025 0 0004
              LDL 3                               0029 A
              EQU                                 0038 F
              BTP A03                             0039 K 0097
     EDIT( X*X * 10 + 1, '*') ., PRINT ., X + 0.1 = X
              LD  X                               0043 0 0004
              LD  X                               0047 0 0004
              MLT                                 0051 E
              LDL 10                              0052 A
              MLT                                 0061 E
              LDL 1                               0062 A
              ADD                                 0071 C
              EDT 01'*'                           0072 I
              PNT                                 0074 0
              LD  X                               0075 0 0004
              LDL 0.1                             0079 A
              ADD                                 0088 C
              ST  X                               0089 B 0004
     .END
              B   A02                             0093 G 0025
       A03                                        0097
.END
              HLT                                 0097 J
              SP  1                               0098
              END                                 0099
\end{verbatim}
\end{figure}




\begin{figure}
\caption{Output from the VALGOL I program given in Figure \ref{valgol-1-example}}
\begin{verbatim}
*
*
*
 *
  *
   *
    *
     *
       *
         *
           *
             *
                *
                   *
                      *
                         *
                            *
                               *
                                   *
                                       *
                                           *
                                               *
                                                    *
                                                         *
                                                              *
                                                                   *
\end{verbatim}
\end{figure}


\begin{figure}
\caption{The META II compiler written in its own language}
\begin{verbatim}
.SYNTAX PROGRAM

OUT1 = '*1' .OUT('GN1') / '*2' .OUT('GN2') /
'*' .OUT('CI') / .STRING .OUT('CL ' *).,

OUTPUT = ('.OUT' '('
$ OUT1 ')' / '.LABEL' .OUT('LB') OUT1) .OUT('OUT') .,

EX3 = .ID .OUT ('CLL' *) / .STRING
.OUT('TST' *) / '.ID' .OUT('ID') /
'.NUMBER' .OUT('NUM') /
'.STRING' .OUT('SR') / '(' EX1 ')' /
'.EMPTY' .OUT('SET') /
'$' .LABEL *1 EX3
.OUT ('BT ' *1) .OUT('SET').,

EX2 = (EX3 .OUT('BF ' *1) / OUTPUT)
$(EX3 .OUT('BE') / OUTPUT)
.LABEL *1 .,

EX1 = EX2 $('/' .OUT('BT ' *1) EX2 )
.LABEL *1 .,

ST = .ID .LABEL * '=' EX1
'.,' .OUT('R').,

PROGRAM = '.SYNTAX' .ID .OUT('ADR' *)
$ ST '.END' .OUT('END').,

.END
\end{verbatim}
\end{figure}


\begin{figure}[htb]
  \caption{Order list of the META II machine}
  \begin{subfigure}[b]{\textwidth}
    \subcaption{MACHINE CODES}
    \begin{tabular}{lllp{7cm}}
      TST & STRING & TEST & After deleting initial blanks in the input
                            string, compare it to the string given as
                            argument. If the comparison is met, delete
                            matched portion from the input and set
                            switch. If not met, reset switch. \\
      ID & & IDENTIFIER & After deleting initial blanks in the input
                          string, test if it begins with an identifier,
                          ie., a letter followed by a sequence of letters
                          and/or digits. If so, delete the identifier and
                          set switch. If not, reset switch. \\
      NUM & & NUMBER & After deleting initial blanks in the input string,
                       test if it begins with a number. a number is a
                       string of digits which may contain imbedded
                       periods, but may not begin or end with a
                       period. moreover, no two periods may be next to one
                       another. If a number is found, delete it and set
                       switch. If not, reset switch. \\
      SR & & STRING & After deleting initial blanks in the input string,
                      test if it begins with a string, ie., a single quote
                      followed by a sequence of any characters other than
                      single quote followed by another single quote. If a
                      string is found, delete it and set switch. if not,
                      reset switch. \\
      CLL & AAA & CALL & Enter the subroutine beginning in location
                         AAA. If the top two terms of the stack are blank,
                         push the stack down by one cell. Otherwise, push
                         it down by three cells. Set a flag in the stack
                         to indicate whether it has been pushed by one or
                         three cells. This flag and the exit address go
                         into the third cell. Clear the top two cells to
                         blanks to indicate that they can accept addresses
                         which may be generated within the subroutine. \\
    \end{tabular}
  \end{subfigure}
\end{figure}
\begin{figure}[htb]\ContinuedFloat
  \begin{subfigure}[b]{\textwidth}
    \begin{tabular}{lllp{7cm}}
      R & & RETURN & Return to the exit address, popping up the stack by
                     one or three cells according to the flag. If the
                     stack is popped by only one cell, then clear the top
                     two cells to blanks, because they were blank when the
                     subroutine was entered. \\
      SET & & SET & Set branch switch on. \\
      B & AAA & BRANCH & Branch unconditionally to location AAA. \\
      BT & AAA & BRANCH IF TRUE & Branch to location AAA if switch is
                                  on. Otherwise, continue in sequence. \\
      BF & AAA & BRANCH IF FALSE & Branch to location AAA if switch is
                                   off. Otherwise, continue in
                                   sequence. \\
      BE & & BRANCH TO ERROR IF FALSE & Halt if switch is off. Otherwise,
                                        continue in sequence. \\
      CL & STRING & COPY LITERAL & Output the variable length string given
                                   as the argument. A blank character will
                                   be inserted in the output following the
                                   string. \\
      CI & & COPY INPUT & Output the last sequence of characters from the
                          input string. This command may not function
                          properly if the last command which could cause
                          deletion failed to do so. \\
      GN1 & & GENERATE 1 & This concerns the current label 1 cell, ie.,
                           the next to top cell in the stack, which is
                           either clear or contains a generated label. If
                           clear, generate a label and put it into that
                           cell. Whether the label has just been put into
                           the cell or was already there, output
                           it. Finally, insert a blank character in the
                           output following the label. \\
      GN2 & & GENERATE 2 & Same as GN1, except that it concerns the
                           current label 2 cell, ie., the top cell in the
                           stack. \\
      LB & & LABEL & Set the output counter to card column 1. \\
      OUT & & OUTPUT & Punch card and reset output counter to card column
                       8. \\
    \end{tabular}
  \end{subfigure}
  \begin{subfigure}[b]{\textwidth}
    \subcaption{CONSTANT AND CONTROL CODES}
    \begin{tabular}{lllp{7cm}}
      ADR & IDENT & ADDRESS & Produces the address which is assigned to
                              the given identifier as a constant. \\
      END & & END & Denotes the end of the program. \\
    \end{tabular}
  \end{subfigure}
\end{figure}


\begin{figure}
\caption{VALGOL II compiler written in META II}
\begin{verbatim}
.SYNTAX PROGRAM

ARRAYPART = '(.' EXP '.)' .OUT('AIA') .,

CALLPART = '(' .OUT('LDF') (EXP $(',' EXP) /
     .EMPTY) ')' .OUT('CLL') .,

VARIABLE = .ID .OUT('LD ' *) (ARRAYPART / .EMPTY) .,

PRIMARY = 'WHOLE' '(' EXP ')' .OUT('WHL') /
     .ID .OUT('LD ' *) (ARRAYPART / CALLPART / .EMPTY) /
     '.TRUE' .OUT('SET') / '.FALSE' .OUT('RST') /
     'O ' .OUT('RST') / '1 ' .OUT('SET') /
     .NUMBER .OUT('LDL' *) /
     '(' EXP ')' .,

TERM = PRIMARY $ ('*' PRIMARY .OUT('MLT') /
     '/' PRIMARY .OUT('DIV') /
     './.' PRIMARY .OUT('DIV') .OUT('WHL') ) .,

EXP2 = '-' TERM .OUT('NEG') /
     '+' TERM / TERM .,

EXP1 = EXP2 $('+' TERM .OUT('ADD') /
     '-' TERM .OUT('SUB')) .,

RELATION = EXP1 (
     '.L=' EXP1 .OUT('LEQ') /
     '.L' EXP1 .OUT('LES') /
     '.=' EXP1 .OUT('EQU') /
     '.-=' EXP1 .OUT('EQU') .OUT('NOT') /
     '.G=' EXP1 .OUT('LES') .OUT('NOT') /
     '.G' EXP1 .OUT('LEQ') .OUT('NOT') /
     .EMPTY) .,

BPRIMARY = '.-' RELATION .OUT('NOT') /
     RELATION .,

BTERM = BPRIMARY $ ('.,' .OUT('BF ' *1)
     .OUT('POP') BPRIAMRY)
     .LABEL *1 .,

BEXP1 = BTERM $( '.V' .OUT('BT ' *1)
     .OUT('POP') BTERM)
     .LABEL *1 .,

IMPLICATION1 = '.IMP' .OUT('NOT')
     .OUT('BT ' *1) .OUT('POP')
     BEXP1 .LABEL *1 .,

IMPLICATION = BEXP1 $ IMPLICATION1 .,

EQUIV = IMPLICATION $('.EQ' .OUT('EQU') ) .,

EXP = '.IF' EXP '.THEN' .OUT('BFP' *1)
     EXP .OUT('B  ' *2) .LABEL *1
     '.ELSE' EXP .LABEL *2 /
     EQUIV .,

ASSIGNPART = '=' EXP ( ASSIGNPART .OUT('ST') /
     ASSIGNPART / (CALLPART / .EMPTY
     .OUT('LDF') .OUT('CLL') )
     .OUT('POP') ) .,

UNTILST = '.UNTIL' .LABEL *1 EXP
     '.DO' .OUT('BTP' *2) ST
     .OUT('B  ' *1) .LABEL *2 .,

WHILECLAUSE = '.WHILE' .OUT('BF ' *1)
     .OUT('POP') EXP .LABEL *1 / .EMPTY .,

FORCLAUSE = VARIABLE '=' .OUT('FLP')
     .OUT('BFP' *1) EXP '.STEP'
     .OUT('SST') .OUT('B  ' *2)
     .LABEL *1 EXP '.UNTIL' .OUT('ADS')
     .LABEL *2 .OUT('RSR') EXP
     .OUT('LEQ') WHILECLAUSE '.DO' .,

FORST = '.FOR' .OUT('SET') .LABEL *1
     FORCLAUSE .OUT('BFP' *2) ST
     .OUT('RST') .OUT('B  ' *1)
     .LABEL *2 .,

IOCALL = 'READ' '(' VARIABLE ',' EXP ')' .OUT('RED') /
     'WRITE' '(' VARIABLE ',' EXP ')' .OUT('WRT') /
     'EDIT' '(' EXP ',' .STRING
     .OUT('EDT' *) ')'
     'PRINT' .OUT('PNT') /
     'EJECT' .OUT('EJT') .,

IDSEQ1 = .ID .LABEL* .OUT('BLK 1') .,

IDSEQ = IDSEQ1 $(',' IDSEQ1) .,

TYPEDEC = '.REAL' IDSEQ .,

ARRAY1 = .ID .LABEL * '(.' 'O' '..' .NUMBER
     .OUT('BLK 1') .OUT('BLK' *) '.)' .,

ARRAYDEC = '.ARRAY' ARRAY1 $( ',' ARRAY1) .,

PROCEDURE = '.PROCEDURE' .ID .LABEL *
     .LABEL *1 .OUT('BLK 1') '('
     (IDSEQ / .EMPTY) ')' .OUT('SP  1') '.,'
     ST .OUT('R  ' *1) .,

DEC = TYPEDEC / ARRAYDEC / PROCEDURE .,

BLOCK = '.BEGIN' .OUT('B  ' *1) $(DEC '.,')
     .LABEL *1 ST $('.,' ST) '.END'
     (.ID / .EMPTY) .,

UNCONDITIONALST = IOCALL / ASSIGNCALLST /
     BLOCK .,

CONDST = '.IF' EXP '.THEN' .OUT('BFP' *1)
     (UNCONDITIONALST ('.ELSE' .OUT('B  ' *2)
     .LABEL *1 ST .LABEL *2 / .EMTPY
     .LABEL *1) / (FORST / UNTILST)
     .LABEL *1) .,

ST = CONDST / UNCONDITIONALST / FORST /
     UNTILST / .EMPTY .,

PROGRAM = BLOCK
     .OUT('HLT' .OUT('SP  1') .OUT('END') .,

.END
\end{verbatim}
\end{figure}


% TODO: Figure (Order list of the VALGOL II machine)

\begin{figure}
\caption{Example program in VALGOL II}
\begin{verbatim}
.BEGIN
.PROCEDURE DETERMINANT(A, N) .,
.BEGIN
.PROCEDURE DUMP() .,
.BEGIN
.REAL D .,
.FOR D = 0 .STEP 1 .UNTIL N-1 .DO
     WRITE(MATRIX(. N*D .), N) .,
PRINT
.END DUMP .,
.PROCEDURE ABS(X) .,
     ABS = .IF 0 .L= X .THEN X .ELSE -X .,
.REAL PRODUCT, FACTOR, TEMP, R, I, J .,
PRODUCT = 1 .,
.FOR R = 0 .STEP 1 .UNTIL N-2
.WHILE PRODUCT .-= 0 .DO .BEGIN
     I  = R .,
     .FOR J = R+1 .STEP 1 .UNTIL N-1 .DO
          .IF ABS( A(. N*I + R .) ) .L
          ABS( A(. N*J + R .) ) .THEN
               I = J .,
     .IF A(. N*I + R .) .= 0 .THEN
          PRODUCT = 0
     .ELSE
          .IF I .-= R .THEN .BEGIN
               PRODUCT = -PRODUCT .,
               .FOR J = R .STEP 1 .UNTIL N-1 .DO
               .BEGIN
                    TEMP = A(. N*R + J .) .,
                    A(. N*R + J .) = A (. N* I + J .) .,
                    A(. N*I + J .) = TEMP .END .END .,
          TEMP = A(. N*R + R .) .,
          .FOR I = R+1 .STEP 1 .UNTIL N-1 .DO
          .BEGIN
               FACTOR = A(. N*I + R .) / TEMP .,
               .FOR J = R .STEP 1 .UNTIL N-1 .DO
                    A(. N*I + J .) = A(. N*I + J .)
                    -FACTOR * A(. N*R + J .) .,
          DUMP
          .END .END .,
.FOR I = 0 .STEP 1 .UNTIL N-1 .DO
     PRODUCT = PRODUCT * A(. N*I + I .) .,
DETERMINANT = PRODUCT
.END DETERMINANT .,
.REAL M, W, T ., .ARRAY MATRIX (. 0 .. 24 .) .,
.UNTIL .FALSE .DO .BEGIN
     EDIT(I, 'FIND DETERMINANT OF' ) ., PRINT., PRINT.,
     READ(M, 1) .,
     .FOR W = 0 .STEP 1 .UTNIL M-1 .DO .BEGIN
          READ(MATRIX (. M*W .), M) .,
          WRITE(MATRIX (. M*W .), M) .END .,
     PRINT .,  T = DETERMINANT (MATRIX, M) .,
     WRITE(T, 1) ., PRINT., PRINT .END
.END PROGRAM
\end{verbatim}


\end{document}
